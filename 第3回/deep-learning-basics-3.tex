\documentclass[dvipdfmx,aspectratio=169]{beamer}
\usepackage{pxjahyper}							%しおりの文字化けを防ぐ
\renewcommand{\kanjifamilydefault}{\gtdefault}	%日本語フォントをゴシックに
\usepackage{graphics}							%各種画像の張り込み
\usepackage{amsmath,amssymb,mathtools}					%標準数式表現を拡大する
\usepackage{ulem}
\usepackage{ascmac,fancybox}
\usetheme[
	block=fill,
	progressbar=foot,
	numbering=fraction,
	subsectionpage=progressbar
]{Metropolis}
\usefonttheme{professionalfonts}

\usepackage{here}
\usepackage{booktabs}

\usepackage{tikz}
\usetikzlibrary{positioning}
\usepackage{color}

\newcommand{\highlight}[2][yellow]{\tikz[baseline=(x.base)]{\node[rectangle,rounded corners,fill=#1!10](x){#2};}}
\newcommand{\highlightcap}[3][yellow]{\tikz[baseline=(x.base)]{\node[rectangle,rounded corners,fill=#1!10](x){#2} node[below of=x, color=#1]{#3};}}

\title{ディープラーニングの仕組みを知ろう!}
\subtitle{第3回 人工知能勉強会}
\author{Shion MORISHITA}
\institute{}
\date{\today}

\subject{\LaTeX{}+Beamer}
\begin{document}
	%タイトル
	\begin{frame}[plain]
	    \maketitle
	\end{frame}
		
	\begin{frame}[shrink]{目次}
		\vspace{1em}
		\tableofcontents
	\end{frame}
	
	\section{はじめに}
	\begin{frame}{目的}
		\begin{itemize}
			\item
		\end{itemize}
	\end{frame}
	
	\section{勾配降下法の復習}
	\begin{frame}{勾配降下法のイメージ}
		\underline{ボールが転がる方向に向かってパラメータ(ここでは$ x $)を更新}
		
		\begin{figure}
			\centering
			\includegraphics[width=0.7\linewidth]{img/image-of-a-ball-rolling-down-a-slope}
		\end{figure}
		
	\end{frame}
	\begin{frame}{ニューラルネットワークのパラメータ}
		\underline{$ w^2_{11}, \dots, w^3_{11}, \dots, b^2_1, \dots, b^3_1, \dots $がパラメータ}
		\begin{figure}
			\centering
			\includegraphics[width=0.8\linewidth]{img/illustration-of-variable-and-parameter-names}
		\end{figure}
	\end{frame}
	\begin{frame}{ニューラルネットワークへの勾配降下法の適用}
		$ \begin{bmatrix}
			\Delta w^2_{11}\\ \vdots\\
			\Delta w^3_{11}\\ \vdots\\
			\Delta b^2_1\\ \vdots\\
			\Delta b^3_1\\ \vdots
		\end{bmatrix} = -\eta \begin{bmatrix}
			\dfrac{\partial C_\mathrm{T}}{\partial w^2_{11}}\\ \vdots\\
			\dfrac{\partial C_\mathrm{T}}{\partial w^3_{11}}\\ \vdots\\
			\dfrac{\partial C_\mathrm{T}}{\partial b^2_1}\\ \vdots\\
			\dfrac{\partial C_\mathrm{T}}{\partial b^3_1}\\ \vdots\\
		\end{bmatrix} $を用いて、$ \begin{bmatrix}
			w^2_{11}\\ \vdots\\
			w^3_{11}\\ \vdots\\
			b^2_1\\ \vdots\\
			b^3_1\\ \vdots
		\end{bmatrix} $を$ \begin{bmatrix}
			w^2_{11} + \Delta w^2_{11}\\ \vdots\\
			w^3_{11} + \Delta w^3_{11}\\ \vdots\\
			b^2_1 + \Delta b^2_1\\ \vdots\\
			b^3_1 + \Delta b^3_1\\ \vdots
		\end{bmatrix} $へ更新を繰り返す。
		
		※正の小さな定数$ \eta $を\alert{学習係数}といい、モデル作成者が自由に設定する。
	\end{frame}
	\begin{frame}{勾配降下法の問題点}
		\underline{微分を実際に計算するのは大変}
		\begin{align*}
			\dfrac{\partial C_\mathrm{T}}{\partial w^2_{11}} 
			&= \sum_{k=1}^{64} \dfrac{\partial C_k}{\partial w^2_{11}}\\
			&= \sum_{k=1}^{64}\left\{ \dfrac{\partial C_k}{\partial a^3_1[k]}\dfrac{\partial a^3_1[k]}{\partial z^3_1[k]}\dfrac{\partial z^3_1[k]}{\partial a^2_1[k]}\dfrac{\partial a^2_1[k]}{\partial z^2_1[k]}\dfrac{\partial z^2_1[k]}{\partial w^2_{11}} + \dfrac{\partial C_k}{\partial a^3_2[k]}\dfrac{\partial a^3_2[k]}{\partial z^3_2[k]}\dfrac{\partial z^3_2[k]}{\partial a^2_1[k]}\dfrac{\partial a^2_1[k]}{\partial z^2_1[k]}\dfrac{\partial a^2_1[k]}{\partial w^2_{11}} \right\}.
		\end{align*}
	\end{frame}
	\begin{frame}{微分地獄の解決策}
		\underline{\alert{誤差逆伝播法} の導入}
		% TODO: \usepackage{graphicx} required
		\begin{figure}
			\centering
			\includegraphics[width=0.9\linewidth]{img/positioning-of-the-error-back-propagation-method}
		\end{figure}
	\end{frame}
	
	\section{誤差逆伝播法}
	\subsection{ユニットの誤差}
	\begin{frame}{誤差逆伝播法とは?}
		\underline{特徴}
		\begin{itemize}
			\item 煩雑な微分計算を、\alert{数列の漸化式}に置き換える
			\item \alert{ユニットの誤差}(error)と呼ばれる変数$ \delta^l_j $を用いる
		\end{itemize}
	\end{frame}
	\begin{frame}{ユニットの誤差$ \delta^l_j $の導入}
		\begin{screen}
			\alert{ユニットの誤差}$ \delta^l_j $を次のように定義する:
			\begin{equation}\label{eq:unit-error}
				\delta^l_j \triangleq \dfrac{\partial C}{\partial z^l_j}
			\end{equation}
		\end{screen}
		なぜこれを導入したか?
		\begin{itemize}
			\item 微分の計算から漸化式の計算へ変えることができるから
		\end{itemize}		
	\end{frame}
	\begin{frame}{ユニットの誤差$ \delta^l_j $のイメージ}
		% TODO: \usepackage{graphicx} required
		\begin{figure}
			\centering
			\includegraphics[width=0.7\linewidth]{img/image-of-unit-error}
		\end{figure}
	\end{frame}
	\begin{frame}[shrink]{重み・バイアスに関する2乗誤差の偏微分を$ \delta^l_j $で表現}
		\begin{center}
			\underline{2乗誤差の重みやバイアスに関する偏微分は、$ \delta^l_j $と密接な関係で結ばれている}
		\end{center}
		
		\underline{例} $ \dfrac{\partial C}{\partial w^2_{11}} $を$ \delta^l_j $で表してみる。
		
		偏微分の連鎖律より、
		\begin{equation}\label{eq:calculate-the-partial-derivative-of-the-squared-error-with-respect-to-the-weights-by-the-chain-rule}
			\dfrac{\partial C}{\partial w^2_{11}} = \dfrac{\partial C}{\partial z^2_1}\dfrac{\partial z^2_1}{\partial w^2_{11}}.
		\end{equation}
		$ z^2_1 $は
		\begin{equation*}
			z^2_1 = w^2_{11}x_1 + \cdots + w^2_{1,12}x_{12} + b^2_1
		\end{equation*}
		と表されるので、
		\begin{equation}\label{eq:partial-derivative-of-z^2_1}
			\dfrac{\partial z^2_1}{\partial w^2_{11}} = x_1.
		\end{equation}
		式\eqref{eq:calculate-the-partial-derivative-of-the-squared-error-with-respect-to-the-weights-by-the-chain-rule}と式\eqref{eq:partial-derivative-of-z^2_1}より、
		\begin{equation}\label{eq:denote-C-as-partial-derivative-of-w^2_11-in-unit-error}
			\dfrac{\partial C}{\partial w^2_{11}} = \delta^2_1 x_1.
		\end{equation}
	\end{frame}
	\begin{frame}{重み・バイアスに関する2乗誤差の偏微分を$ \delta^l_j $で表現(イメージ)}
		% TODO: \usepackage{graphicx} required
		\begin{figure}
			\centering
			\includegraphics[width=0.9\linewidth]{img/image-of-the-partial-derivative-of-the-squared-error-with-respect-to-weights}
		\end{figure}
		
	\end{frame}
\end{document}
